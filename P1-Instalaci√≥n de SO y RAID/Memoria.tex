%%%%%%%%%%%%%%%%%%%%%%%%%%%%%%%%%%%%%%%%%
% Short Sectioned Assignment LaTeX Template Version 1.0 (5/5/12)
% This template has been downloaded from: http://www.LaTeXTemplates.com
% Original author:  Frits Wenneker (http://www.howtotex.com)
% License: CC BY-NC-SA 3.0 (http://creativecommons.org/licenses/by-nc-sa/3.0/)
%%%%%%%%%%%%%%%%%%%%%%%%%%%%%%%%%%%%%%%%%

%----------------------------------------------------------------------------------------
%	PACKAGES AND OTHER DOCUMENT CONFIGURATIONS
%----------------------------------------------------------------------------------------

\documentclass[paper=a4, fontsize=11pt]{scrartcl} % A4 paper and 11pt font size

% ---- Entrada y salida de texto -----

\usepackage[T1]{fontenc} % Use 8-bit encoding that has 256 glyphs
\usepackage[utf8]{inputenc}
%\usepackage{fourier} % Use the Adobe Utopia font for the document - comment this line to return to the LaTeX default

% ---- Idioma --------

\usepackage[spanish, es-tabla]{babel} % Selecciona el español para palabras introducidas automáticamente, p.ej. "septiembre" en la fecha y especifica que se use la palabra Tabla en vez de Cuadro

% ---- Otros paquetes ----

% Hipervínculos
\usepackage[hidelinks]{hyperref}

\usepackage{amsmath,amsfonts,amsthm} % Math packages
%\usepackage{graphics,graphicx, floatrow} %para incluir imágenes y notas en las imágenes
\usepackage{graphics,graphicx, float, url} %para incluir imágenes y colocarlas

% Para hacer tablas comlejas
%\usepackage{multirow}
%\usepackage{threeparttable}

%\usepackage{sectsty} % Allows customizing section commands
%\allsectionsfont{\centering \normalfont\scshape} % Make all sections centered, the default font and small caps

\usepackage{fancyhdr} % Custom headers and footers
\pagestyle{fancyplain} % Makes all pages in the document conform to the custom headers and footers
\fancyhead{} % No page header - if you want one, create it in the same way as the footers below
\fancyfoot[L]{} % Empty left footer
\fancyfoot[C]{} % Empty center footer
\fancyfoot[R]{\thepage} % Page numbering for right footer
\renewcommand{\headrulewidth}{0pt} % Remove header underlines
\renewcommand{\footrulewidth}{0pt} % Remove footer underlines
\setlength{\headheight}{13.6pt} % Customize the height of the header

\numberwithin{equation}{section} % Number equations within sections (i.e. 1.1, 1.2, 2.1, 2.2 instead of 1, 2, 3, 4)
\numberwithin{figure}{section} % Number figures within sections (i.e. 1.1, 1.2, 2.1, 2.2 instead of 1, 2, 3, 4)
\numberwithin{table}{section} % Number tables within sections (i.e. 1.1, 1.2, 2.1, 2.2 instead of 1, 2, 3, 4)

\setlength\parindent{0pt} % Removes all indentation from paragraphs - comment this line for an assignment with lots of text

\newcommand{\horrule}[1]{\rule{\linewidth}{#1}} % Create horizontal rule command with 1 argument of height

%----------------------------------------------------------------------------------------
%	TÍTULO Y DATOS DEL ALUMNO
%----------------------------------------------------------------------------------------

\title{	
\normalfont \normalsize 
\textsc{{\bf Ingeniería de Servidores (2015-2016)} \\ Doble Grado en Ingeniería Informática y Matemáticas \\ Universidad de Granada} \\ [25pt] % Your university, school and/or department name(s)
\horrule{0.5pt} \\[0.4cm] % Thin top horizontal rule
\huge Memoria Práctica 1 \\ % The assignment title
\horrule{2pt} \\[0.5cm] % Thick bottom horizontal rule
}

\author{Óscar Bermúdez Garrido\\ \href{http://www.github.com/oxcar103}{@oxcar103}} % Nombre y apellidos

\date{\normalsize\today} % Incluye la fecha actual

%----------------------------------------------------------------------------------------
% DOCUMENTO
%----------------------------------------------------------------------------------------

\begin{document}

\maketitle % Muestra el Título

\newpage %inserta un salto de página

\tableofcontents % para generar el índice de contenidos

\listoffigures

\listoftables

\newpage


\begin{enumerate}
	\section{Introducción}
	\subsection{Concepto de Máquina Virtual y virtualización}
		\item ¿Qué modos y/o tipos de $"$Virtualización Hardware$"$ existen?(No más de 3 párrafos)
		\cite{Virt}
		
		\item Muestre los precios y características de varios proveedores de VPS(Virtual Private
		Server) y compare con el precio de servidores dedicados(administrados y no administrados)
		de características similares. Comente diferencias.
		
		\item ¿Qué otros software de virtualización existen además de VMWare y Virtual Box?
		
	\section{Instalación de Sistemas Operativos virtualizados}
		\item Enumere algunas de las innovaciones en Windows 2012 R2 respecto a 2008R2.
		
		\item ¿Qué empresa hay detrás de Ubuntu?¿Qué otros productos/servicios ofrece?
		
		El sistema operativo Ubuntu\cite{Ubuntu} es desarrollado por la comunidad de Ubuntu y por la
		empresa británica Canonical Ltd.\cite{Canonical}.
		
		Esta empresa también desarrollada otros proyectos de software libre como \textit{Mir}, un
		servidor gráfico que se planea que sea el sustituto de \textit{X Windows Server} para Ubuntu,
		\textit{Bazaar}, un sistema de control de versiones(especialmente para python), \textit{Juju},
		una herramienta de organización de servicios y \textit{MAAS}\cite{MAAS} que permite el acceso
		y administración de un conjunto de servidores(principalmente desarrollado para Ubuntu Server)
		de forma similar a como se utilizarían los servicios de cloud utilizando \textit{Juju}.
		
		\item ¿Qué relación tiene CentOS con RedHat y con el proyecto Fedora?
		
		\item Indique qué otros SO se utilizan en servidores y el porcentaje de uso(no olvide poner
		la fuente de donde saca la información y preste atención a la fecha de esta).
		
	\subsection{Linux: Particionamiento del Disco Duro virtual y creación de RAID1}
		\item ¿Qué diferencia hay entre RAID mediante SW y mediante HW?\\
		
		\item \begin{enumerate}
			\item ¿Qué es LVM?
			
			\item ¿Qué ventaja tiene para un servidor de gama baja?
			
			\item Si va a tener un servidor web, ¿le daría un tamaño grande o pequeño a /var?\\
			(\href{http://www.tldp.org/HOWTO/LVM-HOWTO/benefitsoflvmsmall.html}
			{http://www.tldp.org/HOWTO/LVM-HOWTO/benefitsoflvmsmall.html})
			(\href{https://wiki.archlinux.org/index.php/LVM#Introduction}
			{https://wiki.archlinux.org/index.php/LVM$\#$Introduction})\\
		\end{enumerate}
		
		\item ¿Debemos cifrar también el volumen que contiene el espacio para swap? ¿y el volumen
		en el que montaremos /boot?
		
		\item ¿Qué otro tipo de usos de una partición le permite configurar el asistente de instalación?
		¿Cuál es la principal diferencia entre ext4 y ext2?
		
		\item Muestre cómo ha quedado el disco particionado una vez el sistema está instalado(lsblk).
		
		\item \begin{enumerate}
			\item ¿Cómo ha hecho el disco 2 $"$arrancable$"$?
			
			\item ¿Qué hace el comando grub-install?
			
			\item ¿Qué hace el comando dd?
		\end{enumerate}
		
		\item \textbf{Opcional:} Muestre(con capturas de pantalla) cómo ha comprobado que el RAID1
		funciona.
		
	\subsection{Windows Server}
		\item ¿Qué diferencia hay entre Standard y Datacenter?
		
		\item Continúe usted con el proceso de definición de RAID1 para los 2 discos de 50MiB que
		ha creado. Muestre el proceso con capturas de pantalla.
		
	\subsection{Ajuste de parámetros de la Máquina Virtual}
		\item Explique brevemente qué diferencias hay entre los 3 tipos de conexión que permite el
		VMSW para las Mvs: NAT, Host-only y Bridge.
		
	\section{Editores de texto}
		\item \textbf{Opcional:} ¿Qué relación hay entre los atajos de teclado de emacs y los de
		la consola de bash?¿y entre los de vi y las páginas del manual?
		
\end{enumerate}


\newpage
\section{Referencias}
\begin{thebibliography}{10}
\expandafter\ifx\csname url\endcsname\relax
  \def\url#1{\texttt{#1}}\fi
\expandafter\ifx\csname urlprefix\endcsname\relax\def\urlprefix{URL }\fi
\expandafter\ifx\csname href\endcsname\relax
  \def\href#1#2{#2} \def\path#1{#1}\fi

\bibitem{Virt}
http://securitywing.com/types-virtualization-technology/

\bibitem{Ubuntu}
Ubuntu. Wikipedia, the free encyclopedia.\\
  \url{http://en.wikipedia.org/wiki/Ubuntu_(operating_system)}

\bibitem{Canonical}
Canonical. Wikipedia, the free encyclopedia.\\
  \url{http://en.wikipedia.org/wiki/Canonical_(company)}

\bibitem{MAAS}
MAAS, Metal As A Service, the official website.\\
  \url{http://maas.ubuntu.com/docs1.5/}

\end{thebibliography}


\end{document}