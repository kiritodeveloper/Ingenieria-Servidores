%%%%%%%%%%%%%%%%%%%%%%%%%%%%%%%%%%%%%%%%%
% Short Sectioned Assignment LaTeX Template Version 1.0 (5/5/12)
% This template has been downloaded from: http://www.LaTeXTemplates.com
% Original author:  Frits Wenneker (http://www.howtotex.com)
% License: CC BY-NC-SA 3.0 (http://creativecommons.org/licenses/by-nc-sa/3.0/)
%%%%%%%%%%%%%%%%%%%%%%%%%%%%%%%%%%%%%%%%%

%----------------------------------------------------------------------------------------
%	PACKAGES AND OTHER DOCUMENT CONFIGURATIONS
%----------------------------------------------------------------------------------------

\documentclass[paper=a4, fontsize=11pt]{scrartcl} % A4 paper and 11pt font size

% ---- Entrada y salida de texto -----

\usepackage[T1]{fontenc} % Use 8-bit encoding that has 256 glyphs
\usepackage[utf8]{inputenc}

% ---- Idioma --------

\usepackage[spanish, es-tabla]{babel} % Selecciona el español para palabras introducidas automáticamente, p.ej. "septiembre" en la fecha y especifica que se use la palabra Tabla en vez de Cuadro

% ---- Otros paquetes ----

% Hipervínculos
\usepackage[hidelinks]{hyperref}

\usepackage{amsmath,amsfonts,amsthm} % Math packages
\usepackage{graphics,graphicx, float, url} %para incluir imágenes y colocarlas

\usepackage{fancyhdr} % Custom headers and footers
\pagestyle{fancyplain} % Makes all pages in the document conform to the custom headers and footers
\fancyhead{} % No page header - if you want one, create it in the same way as the footers below
\fancyfoot[L]{} % Empty left footer
\fancyfoot[C]{} % Empty center footer
\fancyfoot[R]{\thepage} % Page numbering for right footer
\renewcommand{\headrulewidth}{0pt} % Remove header underlines
\renewcommand{\footrulewidth}{0pt} % Remove footer underlines
\setlength{\headheight}{13.6pt} % Customize the height of the header

\numberwithin{equation}{section} % Number equations within sections (i.e. 1.1, 1.2, 2.1, 2.2 instead of 1, 2, 3, 4)
\numberwithin{figure}{section} % Number figures within sections (i.e. 1.1, 1.2, 2.1, 2.2 instead of 1, 2, 3, 4)
\numberwithin{table}{section} % Number tables within sections (i.e. 1.1, 1.2, 2.1, 2.2 instead of 1, 2, 3, 4)

\setlength\parindent{0pt} % Removes all indentation from paragraphs - comment this line for an assignment with lots of text

\newcommand{\horrule}[1]{\rule{\linewidth}{#1}} % Create horizontal rule command with 1 argument of height

%----------------------------------------------------------------------------------------
%	TÍTULO Y DATOS DEL ALUMNO
%----------------------------------------------------------------------------------------

\title{	
\normalfont \normalsize 
\textsc{{\bf Ingeniería de Servidores (2015-2016)} \\ Doble Grado en Ingeniería Informática y Matemáticas \\ Universidad de Granada} \\ [25pt] % Your university, school and/or department name(s)
\horrule{0.5pt} \\[0.4cm] % Thin top horizontal rule
\huge Memoria Práctica 2 \\ % The assignment title
\horrule{2pt} \\[0.5cm] % Thick bottom horizontal rule
}

\author{Óscar Bermúdez Garrido\\ \href{http://www.github.com/oxcar103}{@oxcar103}} % Nombre y apellidos

\date{\normalsize\today} % Incluye la fecha actual

%----------------------------------------------------------------------------------------
% DOCUMENTO
%----------------------------------------------------------------------------------------

\begin{document}
\maketitle % Muestra el Título
\newpage %inserta un salto de página
\tableofcontents % para generar el índice de contenidos
\listoffigures

\begin{enumerate}
	\section{Instalación de servicios y configuración}
	\subsection{yum}
	\item Liste los argumentos de yum necesarios para instalar, buscar y eliminar paquetes.
	
	Basado en \cite{man_yum}:
	
	\textbf{Para instalar:} \textit{sudo yum install $<$paquete$>$}, por ejemplo: \textit{sudo yum
	install oneko}.
	
	\textbf{Para buscar:} \textit{yum search $<$expresión$>$}, por ejemplo: \textit{yum search sl}.
	
	\textbf{Para eliminar:} \textit{sudo yum remove $<$paquete$>$}, por ejemplo: \textit{sudo yum
	remove bsd-games}.
	
	
	\item ¿Qué ha de hacer para que yum pueda tener acceso a Internet?\footnote{Pista: archivo de
	configuración en /etc, proxy: stargate.ugr.es:3128} ¿Cómo añadimos un nuevo repositorio?
	
	Para hacer que yum pueda acceder a Internet a través de un proxy, debemos modificar el archivo
	\textit{/etc/yum/yum.conf}, con el comando \textit{nano} por ejemplo, y añadir la variable
	\textit{proxy} con el valor del proxy(en este caso, \textit{stargate.ugr.es:3128}) como podemos
	ver en \cite{man_yum.conf}, \cite{CentOS_web} y \cite{foro_Fedora}.
	
	Para añadir nuevos repositorios podemos instalar el paquete \textit{yum-utils} y añadir nuevos
	repositorios usando el comando \textit{yum-config-manager} tal y como nos indica su propio manual
	\cite{man_yum-config-manager}. O podemos incluir directamente la dirección del repositorio en
	\textit{/etc/yum/yum.conf} como se indica en \cite{man_yum.conf}.
	
	\subsection{apt}
	\item Indique el comando para buscar un paquete en un repositorio y el correspondiente para instalarlo.
	
	Basado en \cite{man_apt-cache} y \cite{man_apt-get}:
	
	\textbf{Para buscar:} \textit{apt-cache search $<$expresión$>$}, por ejemplo: \textit{apt-cache
	search sl}.
	
	\textbf{Para instalar:} \textit{sudo apt-get install $<$paquete$>$}, por ejemplo: \textit{sudo apt-get
	install oneko}.
	
	\textbf{Para eliminar:} \textit{sudo apt-get remove $<$paquete$>$}, por ejemplo: \textit{sudo apt-get
	remove bsd-games}.
	
	\item Indique qué ha modificado para que apt para acceder a los servidores de paquetes a través de
	proxy. ¿Cómo añadimos un nuevo repositorio?
	
	Para hacer que apt pueda acceder a Internet a través de un proxy, debemos modificar
	el archivo \textit{/etc/apt/apt.conf}, con el comando \textit{nano} por ejemplo, y
	añadir la variable \textit{http\_proxy} con el valor del proxy(en este caso,
	\textit{stargate.ugr.es:3128}) como podemos ver en \cite{man_apt.conf}.
	
	Para añadir nuevos repositorios podemos usar el comando \textit{add-apt-repository}
	tal y como nos indica su propio manual \cite{man_add-apt-repository}. O podemos
	incluir directamente la dirección del repositorio en \textit{/etc/apt/sources.list}
	ya que en \cite{man_apt} se indica que utilicemos \textit{apt edit-sources} para este
	propósito pero lo único que hace este comando es abrirnos el archivo antes citado para
	su edición.

	\subsection{Windows}
	\subsection{OpenSuse}
	\item ¿Qué gestores utiliza OpenSuse?\footnote{Pista: \url{http://es.opensuse.org/Gestión_de_paquetes}}
	
	\section{Gestión de los cortafuegos(\textit{Firewalls})}
	\section{Instalación del servicio de acceso remoto a la consola(\textit{Secure Shell})}
	\item ¿Qué diferencia hay entre telnet y ssh?
	
	\item ¿Para qué sirve la opción \textbf{-X}? Ejecute remotamente, es decir, desde la máquina
	anfitriona (si tiene Linux) o desde la otra máquina virtual, el comando \textit{gedit} en una sesión
	abierta con \textit{ssh}. ¿Qué ocurre?
	
	\item Muestre la secuencia de comandos y las modificaciones a los archivos correspondientes para
	permitir acceder a la consola remota sin introducir la contraseña\footnote{Pista: ssh-keygen, ssh-copy-id}.

	\item ¿Qué archivo es el que contiene la configuración de sshd? ¿Qué parámetro hay que modificar
	para evitar que el usuario root acceda? Cambie el puerto por defecto y compruebe que puede acceder.

	\item Indique si es necesario reiniciar el servicio. ¿Cómo se reinicia un servicio en Ubuntu?
	¿Y en CentOS? Muestre la secuencia de comandos para hacerlo.
	
	\subsection{Utilidades: screen y terminator}
	\item Instale y pruebe \textit{terminator}. Con \textit{screen}, pruebe su funcionamiento dejando
	sesiones \textit{ssh} abiertas en el servidor y recuperándolas posteriomente.
	
	\subsection{Un poco de seguridad: \textit{fail2ban}}
	\item Instale el servicio y pruebe su funcionamiento.
	
	\section{Administración remota de Windows}
	\section{Instalación de un servidor web básico}
	\subsection{Instalación de Apache + MySQL (o MariaDB) + PHP (o Python) en Linux(\textit{LAMP})}
	\item Muestre los comandos que ha utilizado en Ubuntu Server y en CentOS (aunque en este último puede
	utilizar la GUI; en tal caso, realice capturas de pantalla).
	
	% En Ubuntu, usaremos \textbf{lamp-server^} para instalar LAMP.
	% En CentOS, usaremos \textbf{httpd} para instalar Apache, \textbf{mariadb} para Maria-DB, \textbf{php} y \textbf{php-mysql} para PHP y añadiremos Apache a la lista de ejecuciones durante el arranque con \textit{systemctl enable httpd.service}, ahora lo inicializamos con \textit{systemctl start httpd.service} y ejecutamos además \textit{sudo firewall-cmd --permanent --add-service http} y reiniciamos el servicio con \textit{systemcl restart firewalld.service} para que los cambios surtan efecto.
	
	\item Enumere otros servidores web y las páginas de sus proyectos (mínimo 3 sin considerar Apache,
	IIS ni nginx).
	
	% lightspeed, cherokee, lighttp, tomcat, wildfly, glassfish, jonas.
	% Citar mi trabajo del año pasado.
	
	\subsection{Windows: IIS}
	\item Compruebe que el servicio está funcionando accediendo a la MV a través de la anfitriona.
	% Objetivo:
	% FTP Windows con un fichero no vacío y descargarlo desde Ubuntu
	% Poner las máquinas en Red Nat
	% Asegurarse de que se ven (PING)
	% Crear un usuario en Windows
	% Crear un servidor FTP(permitiendo usuario anterior)
	% Firewall puerto 21
	% Activamos el ping accediendo a Administrative Tools -> Server Management -> Configuration -> Windows Firewal... -> Inbound Rules -> File and Printer Sharing...
	% Activamos el puerto de entrada accediendo a Administrative Tools -> Server Management -> Configuration -> Windows Firewal... -> Inbound Rules -> New Rule -> Port -> 21 -> Permitir -> <Nombre>
	% Reiniciamos el firewall en Administrative Tools -> Server Management -> Configuration -> Windows Firewal... -> Services -> <Lo buscamos y reiniciamos>
	% Creamos un usuario accediendo a Administrative Tools -> Server Management -> Configuration -> Local Users and Groups -> Users
	% Administrative Tools -> Server Management -> Roles -> Web Server(IIS) -> Internet Information Services -> Add FTP Service y lo configuramos como queramos.
	\subsection{Windows y otros servidores web}
	\subsection{Java Servlets}
	\item Realice la instalación de uno de estos dos \textit{web containers} y pruebe su ejecución.
	% Instalamos Tomcat y JBoss, creo que eso ya sé XD
	% sudo apt-get install tomcat7
	
	\subsection{Otro tipo de Bases de datos}
	\item Realice la instalación de MongoDB en alguna de sus máquinas virtuales. Cree una colección de
	documentos y haga una consulta sobre ellos \footnote{\url{http://docs.mongodb.org/manual/installation/}}.
	% sudo apt-get install mongodb para instalar la base de datos
	% Trabajar con Mongo:
	%	use <Nombre de la base de datos>
	%	show dbs para mostrarla
	%
	%	db.<Nombre DB>.insert{
	%		<Categoría no numérica>: "<valor>",
	%		<Categoría numérica>: <valor>,
	%	}
	%
	%	Buscar "how to query mongodb" en DuckDuckGo
	%
	
	\section{Manteniendo de los servicios actualizados}
	\item Muestre un ejemplo de uso del comando \footnote{\url{http://fedoraproject.org/wiki/VMWare}}
	
	\section{Administración web}
	\item Realice la instalación de esta aplicación y pruebe a modificar algún parámetro de algún
	servicio. Muestre las capturas de pantalla pertinentes así como el proceso de instalación.
	% Visitar http://www.webmin.com/deb.html y obedecer o forma alternativa:
	% En /etc/apt/sources.list, escribimos deb http://download.webmin.com/download/repository sarge contrib
	% En /etc/apt/sources.list, escribimos deb http://webmin.mirror.somersettechsolutions.co.uk/repository sarge contrib
	% wget http://www.webmin.com/jcameron-key.asc
	% apt-key add jcameron-key.asc
	% apt-get update
	% apt-get install -y webmin
	
	\item Instale \textit{phpMyAdmin}, indique cómo lo ha realizado y muestre algunas capturas de
	pantalla. Configure PHP para poder importar BD's mayores de \textbf{8MiB} (límite por defecto).
	Indique cómo ha realizado el proceso y muestre capturas de pantalla.
	
	\subsection{Más administradores: Ispconfig, Directadmin, C-Panel, Parallels,\dots}
	\item Visite al menos una de las webs de los software mencionados y pruebe las demos que ofrecen
	realizando capturas de pantalla y comentando qué está realizando.
	
	\section{Automatización de tareas con scripts}
	\subsection{Shells}
	\subsubsection*{Comandos grep, find, awk y sed}
	\item Ejecute los ejemplos de find, grep y escriba el script que haga uso de sed para cambiar la
	configuración de ssh y reiniciar el servicio.
	
	% Visitar https://www.digitalocean.com/community/tutorials/the-basics-of-using-the-sed-stream-editor-to-manipulate-text-in-linux
	% Sugerencia:
	% 	Entrada: sshd_config
	% 	Buscar línea donde se especifica el puerto
	% 	Sustituir el puerto por 103
	% 	Reiniciar el servicio
	
	\item Muestre un ejemplo de uso para awk.
	% Visitar http://linux-es.org/node/31
	% Sugerencia:
	% 	Entrada: sshd_config
	% 	Buscar línea donde se especifica el puerto
	% 	Sustituir el puerto por 103
	% 	Si no aparece, indicarlo
	% 	Reiniciar el servicio
	
	\subsection{PHP}
	\subsection{Python}
	\item Escriba el script para cambiar el acceso a ssh usando PHP o Python.
	
	\subsection{Windows PowerShell}
	\item Abra una consola de \textit{Powershell} y pruebe a parar un programa en ejecución, realice
	capturas de pantalla y comente lo que muestra. También puede realizar otra tarea de su elección.
	% Help para ver comandos disponibles
	% Tasklist para ver los programas en ejecución
	% Taskkill para matar el proceso
	
	\subsection{Más automatización}
	
\end{enumerate}

\newpage
\section{Referencias}
\begin{thebibliography}{10}
\expandafter\ifx\csname url\endcsname\relax
  \def\url#1{\texttt{#1}}\fi
\expandafter\ifx\csname urlprefix\endcsname\relax\def\urlprefix{URL }\fi
\expandafter\ifx\csname href\endcsname\relax
  \def\href#1#2{#2} \def\path#1{#1}\fi

\bibitem{man_yum}
Ubuntu manuals\\
yum(8) - Linux man page\\
\url{http://manpages.ubuntu.com/manpages/xenial/en/man8/yum.8.html}

\bibitem{man_yum.conf}
Ubuntu manuals\\
yum.conf(5) - Linux man page\\
  \url{http://manpages.ubuntu.com/manpages/xenial/en/man5/yum.conf.5.html}

\bibitem{CentOS_web}
CentOS.org\\
10. Using yum with a Proxy Server\\
  \url{https://www.centos.org/docs/5/html/yum/sn-yum-proxy-server.html}

\bibitem{foro_Fedora}
Fedoraforum.org\\
A fedora linux support community\\
  \url{http://forums.fedoraforum.org/showthread.php?t=742}

\bibitem{man_yum-config-manager}
Ubuntu manuals\\
yum-config-manager(1) - Linux man page\\
  \url{http://manpages.ubuntu.com/manpages/xenial/en/man1/yum-config-manager.1.html}

\bibitem{man_apt-cache}
Ubuntu manuals\\
apt-cache(8) - Linux man page\\
  \url{http://manpages.ubuntu.com/manpages/xenial/en/man8/apt-cache.8.html}

\bibitem{man_apt-get}
Ubuntu manuals\\
apt-get(8) - Linux man page\\
  \url{http://manpages.ubuntu.com/manpages/xenial/en/man8/apt-get.8.html}

\bibitem{man_apt.conf}
Ubuntu manuals\\
apt.conf(5) - Linux man page\\
  \url{http://manpages.ubuntu.com/manpages/wily/en/man5/apt.conf.5.html}

\bibitem{man_add-apt-repository}
Ubuntu manuals\\
add-apt-repository\\
  \url{http://manpages.ubuntu.com/manpages/natty/man1/add-apt-repository.1.html}

\bibitem{man_apt}
Ubuntu manuals\\
apt(8) - Linux man page\\
  \url{http://manpages.ubuntu.com/manpages/xenial/en/man8/apt.8.html}

\bibitem{oS_packman}
The openSUSE wiki\\
Package Management\\
  \url{https://en.opensuse.org/Package_management}

\bibitem{oS_YaST}
The openSUSE wiki\\
YaST Software Management\\
  \url{https://en.opensuse.org/Portal:YaST}

\bibitem{oS_zypper}
The openSUSE wiki\\
Zypper\\
  \url{https://en.opensuse.org/Portal:Zypper}

\bibitem{oS_YaST_GitHub}
GitHub - How people build software\\
YaST\\
  \url{https://github.com/yast}

\bibitem{oS_zypper_GitHub}
GitHub - How people build software\\
Zypper - Según su propia descripción: "World's most powerful command line package manager"\\
  \url{https://github.com/openSUSE/zypper}

\bibitem{Telnet}
Telnet. Wikipedia, the free encyclopedia.\\
  \url{https://en.wikipedia.org/wiki/Telnet}

\bibitem{SSH}
Secure Shell. Wikipedia, the free encyclopedia.\\
  \url{https://en.wikipedia.org/wiki/Secure_Shell}

\bibitem{man_SSH}
Ubuntu manuals\\
ssh(1) - Linux man page\\
  \url{http://manpages.ubuntu.com/manpages/xenial/en/man1/ssh.1.html}

\bibitem{SSH_StackOverFlow}
StackOverFlow. \\
How to SSH to a VirtualBox guest externally through a host? \\
  \url{http://stackoverflow.com/questions/5906441/how-to-ssh-to-a-virtualbox-guest-externally-through-a-host}

\bibitem{man_ssh-keygen}
Ubuntu manuals\\
ssh-keygen(1) - Linux man page\\
  \url{http://manpages.ubuntu.com/manpages/xenial/en/man1/ssh-keygen.1.html}

\bibitem{man_ssh-copy-id}
Ubuntu manuals\\
ssh-copy-id(1) - Linux man page\\
  \url{http://manpages.ubuntu.com/manpages/xenial/en/man1/ssh-copy-id.1.html}

\bibitem{man_apropos}
Ubuntu manuals\\
apropos(1) - Linux man page\\
  \url{http://manpages.ubuntu.com/manpages/xenial/en/man1/apropos.1.html}

\bibitem{man_service}
Ubuntu manuals\\
service(8) - Linux man page\\
  \url{http://manpages.ubuntu.com/manpages/xenial/en/man8/service.8.html}

\bibitem{man_systemctl}
Ubuntu manuals\\
systemctl(1) - Linux man page\\
  \url{http://manpages.ubuntu.com/manpages/xenial/en/man1/systemctl.1.html}

\bibitem{man_fail2ban}
Ubuntu manuals\\
fail2ban(1) - Linux man page\\
  \url{http://manpages.ubuntu.com/manpages/xenial/en/man1/fail2ban.1.html}

\bibitem{man_jail.conf}
Ubuntu manuals\\
jail.conf(5) - Linux man page\\
  \url{http://manpages.ubuntu.com/manpages/xenial/en/man1/jail.conf.10.html}

\bibitem{man_sshd_config}
Ubuntu manuals\\
sshd\_config(5) - Linux man page\\
  \url{http://manpages.ubuntu.com/manpages/xenial/en/man5/sshd_config.5.html}

\bibitem{man_nano}
Ubuntu manuals\\
nano(1) - Linux man page\\
  \url{http://manpages.ubuntu.com/manpages/xenial/en/man1/nano.1.html}

\bibitem{TWG}
GitHub - How people build software\\
Análisis comparativo de Tomcat, WildFly y GlassFish\\
  \url{https://github.com/oxcar103/Trabajo-ISE}

\bibitem{TC_official}
\textbf{Apache Tomcat}\\
  \url{http://tomcat.apache.org/}

\bibitem{WF_official}
\textbf{JBossDeveloper}\\
  \url{http://wildfly.org/}\\
  \url{http://jbossas.jboss.org/}\footnote{Antiguo enlace a la página oficial, al parecer, esta página
  ha dejado de estar disponible, luego éste ya no está operativo pero me parece que tiene cierto
  interés histórico.}\\

\bibitem{GF_official}
\textbf{GlassFish}\\
  \url{https://glassfish.java.net/}

\bibitem{GF_install}
\textbf{\textit{Java EE 7 with GlassFish 4 Application Server}}\\
David R. Heffelfinger\\
Ed. Packt Publishing (March 2014)\\
Sec. "1. Getting Started with GlassFish"\\
  \url{http://proquest.safaribooksonline.com/book/programming/java/9781782176886}

\bibitem{TC_install}
\textbf{\textit{Apache Tomcat 7 Essentials}}\\
Tanuj Khare\\
Ed. Packt Publishing (March 2012)\\
Sec. "1. Installation of Tomcat7"\\
  \url{http://proquest.safaribooksonline.com/book/operating-systems-and-server-administration/apache/9781849516624}

\bibitem{TC_download}
\textbf{Apache Tomcat}\\
  \url{http://tomcat.apache.org/download-70.cgi}

\bibitem{TC_StackOverFlow}
\textbf{Stack Overflow}
  \url{http://stackoverflow.com/questions/4756039/how-to-change-the-port-of-tomcat-from-8080-to-80}

\bibitem{WF_install}
\textbf{\textit{WildFly: New Features}}\\
Filippe Costa Spolti\\
Ed. Packt Publishing (May 2014)\\
Sec. "1. Starting with WildFly"\\
  \url{http://proquest.safaribooksonline.com/9781783285891?uicode=goliat} 

\bibitem{WF_download}
\textbf{JBossDeveloper}\\
  \url{http://wildfly.org/downloads/}

\bibitem{WF_solution}
\textbf{Dmitriy Sukharev. IT Blog}\\
  \url{http://sukharevd.net/wildfly-8-installation.html}
  \url{https://gist.github.com/sukharevd/6087988}

\bibitem{man_diff}
Ubuntu manuals\\
diff(1) - Linux man page\\
  \url{http://manpages.ubuntu.com/manpages/xenial/en/man1/diff.1posix.html}

\bibitem{man_patch}
Ubuntu manuals\\
patch(1) - Linux man page\\
  \url{http://manpages.ubuntu.com/manpages/xenial/en/man1/patch.1.html}

\bibitem{webmin}
Webmin.com\\
Using the Webmin APT repository\\
  \url{http://webmin.com/deb.html}

\bibitem{man_php}
Ubuntu manuals\\
php(1) - Linux man page\\
  \url{http://manpages.ubuntu.com/manpages/precise/en/man1/php.1.html}

\bibitem{ispconfig}
ispconfig.org\\
Online demo\\
  \url{http://www.ispconfig.org/ispconfig/online-demo/}

\bibitem{man_find}
Ubuntu manuals\\
find(1) - Linux man page\\
  \url{http://manpages.ubuntu.com/manpages/xenial/en/man1/find.1.html}

\bibitem{man_grep}
Ubuntu manuals\\
grep(1) - Linux man page\\
  \url{http://manpages.ubuntu.com/manpages/xenial/en/man1/grep.1posix.html}

\bibitem{man_sed}
Ubuntu manuals\\
sed(1) - Linux man page\\
  \url{http://manpages.ubuntu.com/manpages/xenial/en/man1/sed.1posix.html}

\bibitem{man_awk}
Ubuntu manuals\\
awk(1) - Linux man page\\
  \url{http://manpages.ubuntu.com/manpages/xenial/en/man1/awk.1plan9.html}

\end{thebibliography}


\end{document}